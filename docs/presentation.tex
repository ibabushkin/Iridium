% !TEX TS-program = pdflatex
% !TEX encoding = UTF-8 Unicode
\documentclass{beamer}
\mode<presentation>
{
  \setbeamercovered{transparent}
}
\usepackage[german]{babel}
\usepackage[utf8]{inputenc}
\usepackage{times}
\usepackage[T1]{fontenc}

\title{Softwareprojekt: Iridium}
\subtitle{ein Analyseframework für kompilierte Programme}
\author{Inokentiy Babushkin}
\institute{Abteigymnasium Brauweiler}
\date{06.05.2015 / Colloquium zur Besonderen Lernleistung}

\AtBeginSubsection[]
{
  \begin{frame}<beamer>{Outline}
    \tableofcontents[currentsection,currentsubsection]
  \end{frame}
}
\begin{document}

\begin{frame}
  \titlepage
\end{frame}

\begin{frame}{Inhalt}
  \tableofcontents
\end{frame}

\section{Grundlagen}

\subsection{Prinzipien}

\begin{frame}{Erweiterbarkeit}
  \begin{itemize}
  \item Modularer Aufbau
  \item Sinnvolles API-Design
  \item Multiple Eingabeformate
  \item Lesbarkeitsorientierter Code
  \end{itemize}
\end{frame}

\begin{frame}{Freie Software}
  \begin{itemize}
  \item Verwendung der GPLv3
  \item Codequalität und Funktionsumfang durch Community unterstützt
  \end{itemize}
\end{frame}

\subsection{Reverse Engineering}

\begin{frame}{Definition hier: Wiederherstellung von Hochsprachencode aus kompilierten Programmen}
  \begin{itemize}
  \item Komplexer und zeitaufändiger Prozess, da
    \begin{itemize}
    \item Software zunehmend komplex
    \item Compiler den Code optimisieren
    \end{itemize}
  \item sehr nützlich
    \begin{itemize}
    \item Malware-Analyse
    \item Wiederherstellung von altem Programmcode
    \end{itemize}
  \end{itemize}
\end{frame}

\begin{frame}{(Halb-)Automatisierte Ansätze}
  \begin{itemize}
  \item Decompiler
  \item Analyseframeworks
  \end{itemize}
\end{frame}

\section{Features}

\subsection{Kontrollflussanlyse}

\begin{frame}{Analyse anhand von Kontrollflussgraphen}
  \begin{itemize}
  \item Schrittbasierter Reduktionsalgorithmus für einzelne Funktionen
  \item Unterscheidung aller wichtigen Hochsprachen-Programmelemente
  \item Strukturierte Darstellung
  \end{itemize}
\end{frame}

\subsection{Analyse von Daten auf dem Stack}

\begin{frame}{Heuristiken zur Untersuchung lokaler Variablen}
  \begin{itemize}
  \item Untersuchung des Stackframes der jeweiligen Funktion
  \item (Teilweise) Erkennung von Arrays, Datentypen und Pointern
  \end{itemize}
\end{frame}

\subsection{Analyse optimisierter Integerdivisionen}

\begin{frame}{Optimisierte Integerdivision}
  \begin{itemize}
  \item Von modernen Compiler bei konstantem Divisor angewendet
  \item Teure Divisions-Instruktion wird durch eine Reihe von Additionen, Shifts und Multiplikationen ersetzt
  \item Für Menschen keine Rückschlüsse auf den Divisor möglich, deswegen Implementation eines entsprechenden Algorithmus
  \end{itemize}
\end{frame}

\section*{Zusammenfassung und Ausblick}

\begin{frame}{Zusammenfassung}

  \begin{itemize}
  \item Reverse-Engineering und Analyse kompilierter Programme sind häufig notwendig,
	allerdings meist auch schwieig und mit enormem Aufwand verbunden.
  \item Gleichzeitig sind die automatisierten Lösungsansätze für diese Aufgaben nicht
	oder nur kaum für interaktive Arbeit geeignet, weswegen mit dem vorliegenden
	Projekt ein entsprechender Prototyp vorgelegt wird.
  \item Noch mehr?
  \end{itemize}
  
  \vskip0pt plus.5fill
  \begin{itemize}
  \item
    Ausblick
    \begin{itemize}
    \item Stabilisierung des Codes
    \item Erweiterung um Eingabemodule
    \item Weitere Features
    \end{itemize}
  \end{itemize}
\end{frame}


\end{document}


